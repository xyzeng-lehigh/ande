% Use 12pt for font size of normal text
\documentclass[12pt,letterpaper]{article}
\usepackage{amssymb,amsmath,amsthm}
\usepackage{MnSymbol}
\usepackage{mathrsfs}
%\usepackage{mathabx}
\usepackage{color,graphicx,pifont}
\usepackage{enumerate}
\usepackage{tikz}
\usepackage{mathtools}
\usepackage{extarrows}
\usepackage{hyperref}
\usepackage{caption}
\usepackage{subcaption}
\usepackage{mdwlist}
\usepackage{comment}
%\usepackage{makeidx}
%\makeindex

\usepackage{cleveref}
\crefname{equation}{}{}
\crefname{enumi}{}{}
\crefname{figure}{Figure}{Figure}
\crefname{subsection}{Subsection}{Subsections}
\crefname{lemma}{Lemma}{Lemma}
\crefname{theorem}{Theorem}{Theorem}
\crefname{proposition}{Proposition}{Proposition}
\crefname{section}{Section}{Section}
\crefname{appendix}{Appendix}{Appendix}
\crefname{definition}{Definition}{Definition}
\crefname{corol}{Corollary}{Corollary}
\crefname{table}{Table}{Table}
\crefname{exercise}{Exercise}{Exercise}

% Page Layout
\voffset=-0.25in	% Default is 1.0 inch
\topmargin=0in
\headheight=0in
\headsep=0in

\hoffset=-0.25in	% Default is 1.0 inch
\marginparsep=0in
\marginparwidth=0in
\oddsidemargin=0in

\textwidth=7in
\textheight=9.5in

%\pagestyle{empty}	% no page number
\pagestyle{plain}	% with page number

\begin{document}

\newcommand{\nexthere}{{\color{red} (NEXT HERE)}}

\newcommand{\bs}[1]{\boldsymbol{#1}}
\newcommand{\oname}[1]{\operatorname{#1}}
\newcommand{\eqdef}{\stackrel{\mathrm{def}}{=\joinrel=}}
\def\abs#1{\left|#1\right|}
\def\labs#1{\big|#1\big|}
\setlength{\unitlength}{1cm}
\setlength{\tabcolsep}{0pt}
\def\nrm#1{\left|\left|#1\right|\right|}
\def\vnrm#1{\Vert #1 \Vert}
\def\ab#1{\langle#1\rangle}
\def\lab#1{\left\langle#1\right\rangle}
\newcommand{\po}[1]{\frac{\partial}{\partial{#1}}}
\newcommand{\poh}[1]{\partial/\partial{#1}}
\newcommand{\pp}[2]{\frac{\partial {#1}}{\partial {#2}}}
\newcommand{\pph}[2]{\partial {#1}/\partial {#2}}
\newcommand{\pop}[2]{\frac{\partial^{#1}}{\partial {#2}^{#1}}}
\newcommand{\poph}[2]{\partial^{#1}/\partial {#2}^{#1}}
\newcommand{\ppp}[3]{\frac{\partial^{#1} {#2}}{\partial {#3}^{#1}}}
\newcommand{\ppph}[3]{\partial^{#1} {#2}/\partial {#3}^{#1}}
\newcommand{\flow}[2]{{\mathcal{F}^{#1}_{#2}}}

\newcommand{\hs}{{\oname{HS}}}
\newcommand{\tr}{{\oname{TR}}}
\newcommand{\fred}{{\oname{Fred}}}
\newcommand{\ind}{{\oname{ind}}}

\newtheorem{prop}{Proposition}
\newtheorem{theorem}{Theorem}
\newtheorem{algorithm}{Algorithm}
\newtheorem{definition}{Definition}
\newtheorem{exercise}{Exercise}
\newtheorem{corol}{Corollary}
\newtheorem{lemma}{Lemma}

\numberwithin{prop}{section}
\numberwithin{theorem}{section}
\numberwithin{algorithm}{section}
\numberwithin{definition}{section}
\numberwithin{equation}{subsection}
\numberwithin{figure}{subsection}
\numberwithin{corol}{section}
\numberwithin{lemma}{section}
\numberwithin{exercise}{subsection}

\newcommand\appsecnums{
    \setcounter{subsection}{0}
    \renewcommand\thesubsection{\thesection.\Alph{subsection}}}
\newcommand\noappsecnums{
    \setcounter{subsection}{0}
    \renewcommand\thesubsection{\thesection.\arabic{subsection}}}
\newenvironment{cproof}
    {\color{blue}\begin{proof}}
    {\end{proof}}

\large
% Title
\begin{center}
  {\bf Usage Manual of \textbf{A}nalysis tool of \textbf{N}umerical methods for \textbf{D}ifferential \textbf{E}quations}
\end{center}

\normalsize
% Authors
\begin{center}
Xianyi Zeng$^1$\\
\end{center}
$^1$Assistant Professor, Department of Mathematics, Lehigh University, Bethlehem, PA 18015, U.S.A. email: xyzeng@lehigh.edu\\
\\
\setcounter{tocdepth}{3}
\tableofcontents
%{\small
%\section*{\small Color convention}
%\begin{itemize*}
%\item {Black texts} are normal texts.
%\item {\color{blue} Blue texts} .
%\item {\color{green} Green texts} .
%\item {\color{cyan} Cyan texts} .
%\item {\color{red} Red texts} .
%\end{itemize*}
%}

\newpage

\section{Background}
\label{sec:bg}
\subsection{Constants}
\label{sec:bg_cnst}
This manual will use many combinatoric constants, which are listed below:
\begin{itemize}
  \item {\it The Harmonic numbers}.
    \begin{equation}\label{eq:bg_cnst_harm}
      H_n = \sum_{k=1}^n\frac{1}{k} = 1 + \frac{1}{2} + \cdots + \frac{1}{n}\;,
    \end{equation}
    here $n$ is a non-negative integer.
    By convention, we set $H_0=0$.
  \item {\it Binomial coefficient}.
    As usual, $n!$ denotes the factorial of a non-negative integer $n$ and we denote the binomial coefficients:
    \begin{equation}\label{eq:bg_cnst_binom}
      C_n^k = \binom{n}{k} = \frac{n!}{k!(n-k)!}\;,\quad 0\le k\le n\;.
    \end{equation}
  \item {\it Normalized binomial coefficients}.
    A frequently used constantin this manual is normalized binomial coefficients by a certain number:
    \begin{equation}\label{eq:bg_cnst_brat}
      C^{l,r}_k = \frac{C_{l+r}^{l+k}}{C_{l+r}^l} = \frac{l!r!}{(l+k)!(r-k)!}\;,\quad -l\le k\le r\;.
    \end{equation}
    Here $l$ and $r$ are non-negative integers.
    
\end{itemize}


\subsection{Finite-difference approximations}
\label{sec:bg_fd}
For finite-difference (FD) approximations to a function $u(x)$, we consider nodal values $u_j\approx u(x_j)$, where $x_j=jh$ and $h>0$ is the uniform cell size.

\subsubsection{FD approximation to first-derivative}
\label{sec:bg_fd_dx}
In this section we seek the approximation:
\begin{equation}\label{eq:bg_fd_dx}
  u_x(x_j) \approx \mathcal{D}_xu_j\eqdef \frac{1}{h}\sum_{k=-l}^r\beta_ku_{j+k}
\end{equation}
with optimal accuracy $p=l+r$.
Here we assume $l,r\ge0$, and it is easy to see that $\mathcal{D}_x$ is at least $p$-th order accurate if and only if for any $P(x)\in\mathbb{P}^p$, there is:
\begin{equation}\label{eq:bg_fd_dx_pexact}
  P'(x_j) = \frac{1}{h}\sum_{k=-l}^rP(x_{j+k})\;,
\end{equation}
which is known as the $p$-exactness condition.
One can show that given the stencil $(l,r)$, the operator $\mathcal{D}_x$ with order $p=l+r$ is unique and optimal, which is denoted $\mathcal{D}_x^{l,r}$.

\medskip

\noindent
\textbf{\textit{Explicit formula of coefficients}}.

\smallskip
The first consequence of~\cref{eq:bg_fd_dx_pexact} is the explicit formula for $\mathcal{D}^{l,r}_x$, as derived next.
Define $L(x)=\prod_{k=-l}^r(x-x_{j+k})\in\mathbb{P}^{p+1}$ and $L_k(x)=L(x)/(x-x_{j+k})\in\mathbb{P}^p$, then the Lagrangian interpolation basis polynomials for the set $\{x_{j+k}:\,-l\le k\le r\}$ are given by $\{\hat{L}_k(x)\eqdef L_k(x)/L_k(x_{j+k}):\,-l\le k\le r\}$.
It follows that for all $P(x)\in\mathbb{P}^p$:
\begin{displaymath}
  P(x) = \sum_{k=-l}^rP(x_{j+k})\hat{L}_k(x)\;,
\end{displaymath}
hence the coefficients in $\mathcal{D}^{l,r}_xu_j=\frac{1}{h}\sum_{k=-l}^r\beta_k^{l,r}u_{j+k}$ is given by:
\begin{equation}\label{eq:bg_fd_dx_coef}
  \beta_k^{l,r} = h\hat{L}_k'(x_j) = \left\{\begin{array}{lcl}
    \frac{(-1)^{k-1}}{k}C_k^{l,r}\,, & & -l\le k\le r,\,k\ne0 \\ \vspace{-.1in} \\
    H_l-H_r\,, & & k=0\;.
  \end{array}\right.
\end{equation}
%Here $H_m=1+\frac{1}{2}+\cdots+\frac{1}{m}$ are Harmonic numbers and by convention we denote $H_0=0$.
When the stencil is obvious from the context, we shall write $\mathcal{D}_x^{l,r}$ and $\beta^{l,r}_k$ as $\mathcal{D}_x$ and $\beta_k$ for short.

\medskip

\noindent
\textbf{\textit{Semi-discretized approximation to linear advection equations}}.

\smallskip
Let us consider the semi-discretization of the linear advection equation $u_t+cu_x=0$, where $c>0$ is a constant.
Following the standard Fourier analysis, we consider a simple wave given by $u(x,t)=e^{i\kappa(x-ct)}$.
The semi-discretized scheme is given by:
\begin{equation}\label{eq:bg_fd_dx_adv_semi}
  u_j' = -\frac{c}{h}\mathcal{D}_xu_j\;.
\end{equation}
On the one hand, the local truncation error to this method is $p$-th order, that is if one substitute $u_j$ with the exact solution, they obtain:
\begin{displaymath}
  -ic\kappa e^{i\kappa(x_j-ct)} = u_t(x_j,t) = -cu_x(x_j,t) = -\frac{c}{h}\sum_{k=-l}^ru(x_{j+k},t) + O(h^{p+1}) = -\frac{c}{h}\sum_{k=-l}^r\beta_k e^{i\kappa(x_{j+k}-ct)} + O(h^{p+1})\;,
\end{displaymath}
which can be simplified to:
\begin{equation}\label{eq:bg_fd_dx_adv_lte}
  i\theta = \sum_{k=-l}^r\beta_ke^{ik\theta}+O(\theta^{p+1})\quad\textrm{or}\quad
  -i\theta e^{-ir\theta} = -\sum_{k=-l}^r\beta_ke^{-i(r-k)\theta} + O(\theta^{p+1})\;,
\end{equation}
where $\theta=\kappa h$.
The accuracy concerns $\theta\approx0$, or $e^{-i\theta}\approx1$; to this end we may define $e^{-i\theta}=1+z$, so that $O(\theta)=O(z)$ and~\cref{eq:bg_fd_dx_adv_lte}$_2$ reads:
\begin{displaymath}
  S_r(z) = (1+z)^r\ln(1+z) = S_{l,r}(z) + R_{l,r}(z)\;,\ S_{l,r}(z) = -\sum_{k=-l}^r\beta_k(1+z)^{r-k}\in\mathbb{P}^p(z),\ R_{l,r}(z)=O(z^{p+1})\;.
\end{displaymath}
It follows that $S_{l,r}(z)$ is the leading $p$-th degree polynomial approximation in the Taylor series of $S_r(z)=(1+z)^r\ln(1+z)$ about $z=0$, and the remainder term $R_{l,r}(z)$ can be obtained by the Taylor theorem in integral form:
\begin{equation}\label{eq:bg_fd_dx_adv_rem}
  R_{l,r}(z) = \int_0^1\frac{(1-t)^pS_r^{(p+1)}(tz)z^{p+1}}{p!}dt\;.
\end{equation}
By direct calculation:
\begin{displaymath}
  S_r^{(p+1)}(z) = \frac{(-1)^{p-r}r!(p-r)!}{(1+z)^{p+1-r}} = \frac{(-1)^lr!l!}{(1+z)^{l+1}}\;.
\end{displaymath}
This gives rise to the following identity:
\begin{displaymath}
  i\theta = -\sum_{k=-l}^r\beta_ke^{-ik\theta}+e^{-ir\theta}\int_0^1\frac{(-1)^l(1-t)^pr!l!z^{p+1}}{p!(1+tz)^{l+1}}dt\;.
\end{displaymath}
In the last integral, the integration path is the straight line from $1$ to $1+z=e^{i\theta}$; and the integrant has a singularity along the path if $z=-1$.
Hence by the Cauchy's integral theorem, for all $0\le\theta<\pi$ we may pick a different path connecting $1$ and $e^{i\theta}$, namely $e^{i\varphi}, 0\le\varphi\le\theta$, and obtain:
\begin{align}
  \notag
  i\theta &= -\sum_{k=-l}^r\beta_ke^{-ik\theta}+e^{-ir\theta}\int_0^{\theta}\frac{(-1)^l(e^{i\theta}-e^{i\varphi})^pr!l!}{p!e^{i(l+1)\varphi}}de^{i\varphi}\;. \\
  \notag
  &= -\sum_{k=-l}^r\beta_ke^{-ik\theta}+\frac{(-1)^lr!l!}{p!}\int_0^{\theta}(e^{i(\theta-\varphi)}-1)^pe^{-ir(\theta-\varphi)}d\varphi \\
  \label{eq:bg_fd_dx_adv_semi_symbol}
  &= -\sum_{k=-l}^r\beta_ke^{-ik\theta}+\frac{(-1)^lr!l!}{p!}\int_0^{\theta}(2i\sin\frac{\varphi}{2})^pe^{\frac{1}{2}(l-r)i\varphi}d\varphi\;.
\end{align}
To study the stability of the semi-discretized system~\cref{eq:bg_fd_dx_adv_semi}, we assume initial data $u(x,0)=e^{i\kappa x}$ then the solution is given by $u_j(t)=A(t)e^{ij\theta}$, where:
\begin{displaymath}
  \frac{A'(t)}{A(t)} = -\frac{c}{h}\sum_{k=-l}^r\beta_ke^{ik\theta} = -\frac{c}{h}\mu(\theta) = -i\frac{c}{h}\omega(\theta)\;.
\end{displaymath}
Here $\omega$ is the numerical wave number and $\omega(\theta)=\theta+O(\theta^{p+1})$ by~\cref{eq:bg_fd_dx_adv_lte}; and stability of the semi-discretized method requires $\oname{Re}\mu(\theta)\ge 0$ for all $0\le \theta\le 2\pi$, and any zero of $\oname{Re}\mu(\theta)$ is simple:
\begin{equation}\label{eq:bg_fd_dx_adv_re}
  \oname{Re}\mu = \sum_{k=-l}^r\beta_k\cos(k\theta)
\end{equation}

Lastly, we describe the stability region and order stars of the semi-discretized methods.
By~\cref{eq:bg_fd_dx_adv_lte}, we have
\begin{equation}\label{eq:bg_fd_dx_lambda_order}
  \sigma(\theta) = \lambda(\theta) - i\theta = O(\theta^{p+1})\;;
\end{equation}
and for stability, one requires $\oname{Re}\lambda\ge0$ or $\oname{Re}\sigma\ge0$ for all $0\le\theta\le2\pi$, and any zero of $\oname{Re}\sigma$ is simple.
Denoting $z=i\theta$ and abusing the notation $\lambda(z)$ and $\sigma(z)$, one defines the stability region of the semi-discretized method:
\begin{equation}\label{eq:bg_fd_dx_lambda_stab}
  \mathcal{S} = \{z\in\mathbb{C}:\,\oname{Re}\sigma(z)\ge0\}\;,
\end{equation}
then a necessary condition (and almost sufficient condition less the simple-root requirement on the imaginary axis) for the method to be stable is $i\mathbb{R}\subseteq\mathcal{S}$.
The order star is defined as the complement of $\mathcal{S}$:
\begin{equation}\label{eq:bg_fd_dx_lambda_os}
  \mathcal{O} = \{z\in\mathbb{C}:\,\oname{Re}\sigma(x)<0\}\;.
\end{equation}
In the plots we will make for the order stars, the regions of $\mathcal{O}$ are indicated by shaded regions and those in the interior of $\mathcal{S}$ are not colored.
In the vicinity of $z=0$, $\sigma(z)=O(z^{p+1})$; thus $z=0$ is adjoined by $p+1$ sectors of $\mathcal{O}$, interlaced by $p+1$ sectors of $\mathcal{S}$, and each sector has the asymptotic angle $\frac{\pi}{p+1}$ as $z\to0$.
This is also the origin of the name ``order star''.


\end{document}
